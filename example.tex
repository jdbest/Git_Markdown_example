% Options for packages loaded elsewhere
\PassOptionsToPackage{unicode}{hyperref}
\PassOptionsToPackage{hyphens}{url}
%
\documentclass[jou]{apa7}

\usepackage{amsmath,amssymb}
\usepackage{lmodern}
\usepackage{iftex}
\ifPDFTeX
  \usepackage[T1]{fontenc}
  \usepackage[utf8]{inputenc}
  \usepackage{textcomp} % provide euro and other symbols
\else % if luatex or xetex
  \usepackage{unicode-math}
  \defaultfontfeatures{Scale=MatchLowercase}
  \defaultfontfeatures[\rmfamily]{Ligatures=TeX,Scale=1}
\fi
% Use upquote if available, for straight quotes in verbatim environments
\IfFileExists{upquote.sty}{\usepackage{upquote}}{}
\IfFileExists{microtype.sty}{% use microtype if available
  \usepackage[]{microtype}
  \UseMicrotypeSet[protrusion]{basicmath} % disable protrusion for tt fonts
}{}
\makeatletter
\@ifundefined{KOMAClassName}{% if non-KOMA class
  \IfFileExists{parskip.sty}{%
    \usepackage{parskip}
  }{% else
    \setlength{\parindent}{0pt}
    \setlength{\parskip}{6pt plus 2pt minus 1pt}}
}{% if KOMA class
  \KOMAoptions{parskip=half}}
\makeatother
\usepackage{xcolor}
\IfFileExists{xurl.sty}{\usepackage{xurl}}{} % add URL line breaks if available
\IfFileExists{bookmark.sty}{\usepackage{bookmark}}{\usepackage{hyperref}}
\hypersetup{
  pdftitle={Example document with some references},
  pdfkeywords={LaTeX, RMarkdown, Document formatting},
  hidelinks,
  pdfcreator={LaTeX via pandoc}}
\urlstyle{same} % disable monospaced font for URLs
\setlength{\emergencystretch}{3em} % prevent overfull lines
\providecommand{\tightlist}{%
  \setlength{\itemsep}{0pt}\setlength{\parskip}{0pt}}
\setcounter{secnumdepth}{-\maxdimen} % remove section numbering
\ifLuaTeX
  \usepackage{selnolig}  % disable illegal ligatures
\fi
\newlength{\cslhangindent}
\setlength{\cslhangindent}{1.5em}
\newlength{\csllabelwidth}
\setlength{\csllabelwidth}{3em}
\newenvironment{CSLReferences}[2] % #1 hanging-ident, #2 entry spacing
 {% don't indent paragraphs
  \setlength{\parindent}{0pt}
  % turn on hanging indent if param 1 is 1
  \ifodd #1 \everypar{\setlength{\hangindent}{\cslhangindent}}\ignorespaces\fi
  % set entry spacing
  \ifnum #2 > 0
  \setlength{\parskip}{#2\baselineskip}
  \fi
 }%
 {}
\usepackage{calc}
\newcommand{\CSLBlock}[1]{#1\hfill\break}
\newcommand{\CSLLeftMargin}[1]{\parbox[t]{\csllabelwidth}{#1}}
\newcommand{\CSLRightInline}[1]{\parbox[t]{\linewidth - \csllabelwidth}{#1}\break}
\newcommand{\CSLIndent}[1]{\hspace{\cslhangindent}#1}

\title{Example document with some references}
\shorttitle{Example R Markdown APA doc}
\leftheader{Dainer-Best et al.}
\authorsnames[{1},{2},{2,3},{1}]{Justin Dainer-Best,Second Author,Third
Author,Fourth Author}
\authorsaffiliations{{Bard College}
,{Other Institution}
,{Yet another Institution}
}
\authornote{This is an author's note; you could remove it if not
desired. You could include a corresponding author, and their email.}
\date{March 29, 2021}
\abstract{This is an example of creating an approximately APA-style
manuscript using R Markdown and .bib files}
\keywords{LaTeX, RMarkdown, Document formatting}




\begin{document}
\maketitle

\hypertarget{introduction}{%
\section{Introduction}\label{introduction}}

This is a simple example of using R Markdown documents to create
APA-formatted documents with LaTeX compiling them into PDFs. You will
need to install a version of LaTeX to compile; some suggestions are
included
\href{https://bookdown.org/yihui/rmarkdown-cookbook/install-latex.html}{here}
which involve using \href{https://yihui.org/tinytex/}{TinyTeX}
(\protect\hyperlink{ref-xie2019}{Xie, 2019}). You may need to install
the \texttt{apa7} LaTeX package
(\protect\hyperlink{ref-weiss2021}{Beitzel, 2021}); I believe you should
be able to do so with \texttt{tinytex::tlmgr\_install("apa7")} (but I
have not yet been able to test this on a machine without a LaTeX
installation).

If TinyTeX does not work for you or if you intend to use LaTeX beyond
for this work, you may want to install
\href{http://tug.org/mactex/}{MacTeX} for Macs or
\href{https://miktex.org/}{MiKTeX} for PCs. (These are large
installations---thus the point of a ``tiny'' version.)

You will also have to compile your references in a bibliography. Most
folks recommend using Zotero, as do I (manually maintaining references
can be frustrating); you'll need to output your references to a .bib
file. Document/article titles will follow the capitalization you've
got---use APA style yourself! (The most recent versions of R Markdown
can also
\href{https://rmarkdown.rstudio.com/authoring_bibliographies_and_citations.html}{help
add your references}, including just from a DOI link.)

Some packages (e.g., \{\href{https://github.com/crsh/papaja}{papaja}\};
(\protect\hyperlink{ref-austbarth2020}{Aust \& Barth, 2020})) will
create APA-draft manuscripts, but may introduce extra details beyond the
basic style. The style I lay out here will simply create a document
using APA formatting of references.

\hypertarget{getting-started}{%
\section{Getting started}\label{getting-started}}

You will need to download the \texttt{apa.csl} and \texttt{template.tex}
documents into a directory where your Rmd file lives. Open (or create)
your R Markdown document.

Your document will automatically include some YAML headers:

\begin{verbatim}
---
title: "Your title"
author: "Your name"
date: "3/29/2021"
output: pdf_document
---
\end{verbatim}

You can keep the title and date, but you will need to update the others
headers to include some of the following YAML headers in your R Markdown
document. In particular, you must include the \texttt{doctype}, the full
set of \texttt{output} parameters
(\texttt{output:\ pdf\_document:\ template:\ "template.tex"}), and the
\texttt{bibliography} and \texttt{csl} lines. The others, including
\texttt{shorttitle} and \texttt{leftheader}, \texttt{authors\_note}, and
so forth, are implemented in APA style documents.

\begin{verbatim}
---
doctype: man # this can be stu, jou, doc, or man
title: "Your title"
author: 
  - name: "Your name"
    affiliation_number: 1
affiliations: "Your affiliation"
shorttitle: "APA short title"
authors_note: |
  This is an author's note; 
  you could remove it if not desired. 
abstract: "Your article's abstract"
keywords: "example, keywords"
date: "3/29/21"
output: 
  pdf_document:
    template: "template.tex"
bibliography: example.bib # update to your .bib file!
csl: apa.csl
---
\end{verbatim}

A minimal example is included (minimal\_example.Rmd and
minimal\_example.pdf) with these headers; feel free to download it and
adapt. Without any changes, it will make the PDF you can see here, so
long as you have installed LaTeX---and downloaded the template.tex and
apa.csl files into the same directory.

If you look at example.Rmd, you'll see how to include multiple authors
with differing affiliations. List each author separately. Including
\texttt{affiliation\_number} will provide a superscript number after
that name, which then anticipates a subsequent item under
\texttt{affiliations}. (If there are three
\texttt{affiliation\_number}s, there should be three listed
\texttt{affiliations}.) If an author has multiple affiliations, simply
enclose the numbers in quotation marks, as below, separated with a
comma: ``2,3.''

\begin{verbatim}
author: 
  - name: "First Author"
    affiliation_number: 1
  - name: "Second Author"
    affiliation_number: 2
  - name: "Third Author"
    affiliation_number: "2,3"
  - name: "Fourth Author"
    affiliation_number: 1
affiliations:
  - "First Institution"
  - "Other Institution"
  - "Yet another Institution"
\end{verbatim}

As described below, there are some additional headers you may include:

\begin{itemize}
\tightlist
\item
  \texttt{leftheader}: The authors' last names (for \texttt{jou} doctype
  only)
\item
  For the \texttt{stu} doctype, info about the course:
  \texttt{professor}, \texttt{course}, and \texttt{duedate}
\end{itemize}

There are additional YAML headers often used for different documents in
R Markdown, many of which may work here.

\hypertarget{document-types}{%
\section{Document types}\label{document-types}}

The \textbf{\{apa7\}} document class in LaTeX accepts four document
types---they're mentioned briefly above under \texttt{doctype}; you can
try each out. Here's what they are:

\begin{itemize}
\tightlist
\item
  \emph{jou}: journal style; intended to mimic what a journal looks like
  (two columns, etc). This will look weird with the minimal\_example
  file---it looks better with a longer file -- this will also make use
  of the YAML \texttt{leftheader} and \texttt{shorttitle}, which will
  alternate page headers.
\item
  \emph{doc}: a one-column PDF output with APA style (including
  abstract, etc); it will try to be the ``easiest to read''
\item
  \emph{man}: APA's suggestion for how to submit documents to a
  journal---what many instructors ask for in college course, with
  double-spacing and so forth
\item
  \emph{stu}: student papers---includes YAML headers \texttt{professor},
  \texttt{course}, and \texttt{duedate}; otherwise much like
  \texttt{man}
\end{itemize}

I've created versions of this Rmd file with each document style, so you
can see what they might look like.

\hypertarget{writing-in-an-r-markdown-file}{%
\section{Writing in an R Markdown
file}\label{writing-in-an-r-markdown-file}}

Now that you have a working R Markdown file, you've chosen a document
type, and you've updated it with your paper title and name, etc., your
focus is on writing the document!

If you've ever looked into LaTeX document writing, this is more simple.
There are three major things to keep in mind:

\begin{enumerate}
\def\labelenumi{\arabic{enumi}.}
\tightlist
\item
  Use \# to start a section, \#\# to start a subsection, etc.
\item
  Use square brackets {[}{]} to enclose references, e.g.,
  \texttt{{[}@referencekey{]}}, where you precede a key with the @
  symbol. The ``reference key'' is the way it will be referred in your
  .bib file---in example.bib, for example, you'll see that it's the
  first word of each entry. For more info on citations,
  \href{https://rmarkdown.rstudio.com/authoring_bibliographies_and_citations.html\#citations}{see
  here}.
\end{enumerate}

Most other formatting will be done for you. If you want to learn things
about formatting in R Markdown, there are a variety of
\href{https://rmarkdown.rstudio.com/lesson-15.html}{cheatsheets}
available; you can refer to the example.Rmd document here to learn about
links, lists, \emph{italics} or \textbf{bold-face text}, etc.

\hypertarget{wrapping-up}{%
\section{Wrapping up}\label{wrapping-up}}

You needn't write anything below the title of your references
section---the bibliography will be automatically generated. If you
notice that something is inappropriately not capitalized, make sure it
is capitalized correctly in your .bib file; if it is, consider
surrounding it with curly brackets to ensure

\hypertarget{references}{%
\section*{References}\label{references}}
\addcontentsline{toc}{section}{References}

\hypertarget{refs}{}
\begin{CSLReferences}{1}{0}
\leavevmode\hypertarget{ref-austbarth2020}{}%
Aust, F., \& Barth, M. (2020). \emph{{papaja}: {Create} {APA}
manuscripts with {R Markdown}}. \url{https://github.com/crsh/papaja}

\leavevmode\hypertarget{ref-weiss2021}{}%
Beitzel, B. D. (2021). \emph{Formatting documents in {APA} style (7th
{E}dition) with the apa7 LaTeX class}.
\url{http://ctan.math.washington.edu/tex-archive/macros/latex/contrib/apa7/apa7.pdf}

\leavevmode\hypertarget{ref-xie2019}{}%
Xie, Y. (2019). TinyTeX: A lightweight, cross-platform, and
easy-to-maintain LaTeX distribution based on TeX live. \emph{TUGboat},
\emph{1}, 30--32.
\url{http://tug.org/TUGboat/Contents/contents40-1.html}

\end{CSLReferences}

\end{document}